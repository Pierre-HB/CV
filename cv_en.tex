\documentclass[11pt,a4paper]{moderncv}
\usepackage{multicol}
% moderncv themes
\moderncvtheme[green]{casual}
% \moderncvtheme[green]{classic}


% character encoding
\usepackage[utf8]{inputenc}
% adjust the page margins
% \usepackage[scale=0.8]{geometry}
\usepackage[top=2cm,bottom=2.5cm,left=2cm,right=2cm]{geometry}
\recomputelengths
% personal data
\firstname{Pierre}
\familyname{Hubert-Brierre}
\title{Computer Graphics PhD Student}
\address{91400 Orsay}{France}
\phone{+33 (0) 6 63 31 76 25}
\email{pierre.hubert-brierre@ens-lyon.fr}
\extrainfo{\url{https://github.com/Pierre-HB}}
% \photo[64pt]{photo_Pierre.jpg}
\begin{document}


\maketitle

% \vspace{1cm}
\section{Publication}
\cventry{2025}{Accelerating Signed Distance Functions}{Pacific Graphics}{}{}{P. Hubert-Brierre, E. Guérin, A. Peytavie, E. Galin}

\vspace{0.5cm}
\section{Certificates}


\cventry{2024}{ENS Certificate}{Computer Science}
{ENS (Lyon)}{Certificate of fundamental research}{}

\cventry{2023}{Master of Science ID3D}{Computer Graphics}
{Claude Bernard University (Lyon)}{}{}

\cventry{2022}{Cambridge Advanced}{English Level C1}
{}{}{}

\cventry{2021}{Honour Degree}{Mathematics}{Claude Bernard University (Lyon)}{}{}
\cventry{2021}{Honour Degree}{Computer Science}{Claude Bernard University (Lyon)}{}{}

\cventry{2018}{Baccalaureate}{S (SI) European with highest honour}
{Jules Ferry High School (Cannes)}{}{}


\vspace{0.5cm}
\section{Teachings}

\cventry{2025}{OpenGL Introduction}{Polytechnique (36h)}
{Teaching Assistant}{}{}
\cventry{2024/2025}{Computer Vision}{Polytechnique (12h+18h)}
{Teaching Assistant}{}{}

\vspace{0.5cm}
\section{Curriculum}


\cvline{2024-}{PhD Student in Computer Science (Claude Bernard University/Polytechnique)}
\cvline{2023-2024}{Ecole Normal Supérieur (Lyon), Computer Science Department}
\cvline{2022-2023}{Claude Bernard University (Lyon), Computer Graphics}
\cvline{2020-2022}{Ecole Normal Supérieur (Lyon), Computer Science Department}
\cvline{2018-2020}{MPSI, MP* School St-Louis (Paris)}{}{}

\vspace{0.5cm}
% \newpage
\section{Internships}

\cventry{2024}{LIX}{Polytechnique}{5 months, with Marie-Paule CANI}
{\newline Implicit surfaces optimisation (Internship preparing my PhD topic).}{}

\cventry{2023}{LIRIS}{Claude Bernard University}{3 months, with Eric GALIN}
{\newline Implicit Surfaces Modeling (Internship preparing my PhD topic).}{}

\cventry{2023}{LIRIS}{Claude Bernard University}{5 months, with Jean-Claude IHEL}
{\newline Defensive sampling for Monte Carlo rendering (Two minor descoveries).}{}

\cventry{2022}{University of Edimburg}{Edimbourg}{3 months, with Kartic SUBR}
{\newline NLP for Computer Graphics (Automatic generation of 3D scenes and a user study).}{}

\cventry{2021}{INRIA}{Bordeaux}{6 weeks, with Pascal BARLA}
{\newline Thin layer modelisation (Software development  for BRDF precomputation)}{}

%stage de 3eme...
%\cventry{2014}{Cochlear}{Bruxelles}{1 semaines, avec Florent HUBERT-BRIERRE}{\newline Introduction à la recherche}{}


\newpage

\section{Personal projects (\href{https://github.com/Pierre-HB}{https://github.com/Pierre-HB})}


\cvitem{2022}{Procedural generation of city and road using C++}
\cvitem{2022}{Fractal tree with Implicit surfaces using C++}
\cvitem{2022}{Ray tracing algorithm to render displacement map using C++}
\cvitem{2020}{Greedy algorithm to solve the Travelling salesman problem with intensive optimisations using C++}
\cvitem{2020}{First ray tracing algorithm using JAVA}
\cvitem{2020}{3D engine using OpenGL with JAVA}
\cvitem{2019}{Improvement of my 3D engine using Python and Ocaml}
\cvitem{2019}{First 3D engine using Python}
\cvitem{2018}{2D game in JAVA using LWJGL}

\vspace{0.5cm}
\section{Languages}


\cvlanguage{French}{Mother tongue}{}
\cvlanguage{English}{C1 level (Cambridge Advanced)}{}


\vspace{0.5cm}
\section{Programming languages}

\vspace{0.2cm}

\cvlistdoubleitem{Python}{C / C++}

\vspace{0.5cm}
\section{Miscellaneous}

\subsection{Interests}


\cvlistdoubleitem{Video games}{Ballroom dance}
\cvlistdoubleitem{Hiking}{Roller}

\vspace{0.2cm}
\subsection{Qualities}


\cvlistdoubleitem{Versatile}{Autonomous}


\end{document}


