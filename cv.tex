\documentclass[11pt,a4paper]{moderncv}
% moderncv themes
\moderncvtheme[green]{casual}
% \moderncvtheme[green]{classic}

% character encoding
\usepackage[utf8]{inputenc}
% adjust the page margins
% \usepackage[scale=0.8]{geometry}
\usepackage[top=2cm,bottom=2.5cm,left=2cm,right=2cm]{geometry}
\recomputelengths
% personal data
\firstname{Pierre}
\familyname{Hubert-Brierre}
\title{étudient en recherche informatique}
\address{69007 Lyon}{France}
\mobile{+33 6 63 31 76 25}
\phone{06 63 31 76 25}
\email{pierre.hubert-brierre@ens-lyon.fr}
\extrainfo{\url{https://github.com/Pierre-HB}}
\photo[64pt]{photo_Pierre.jpg}
\begin{document}
\maketitle


\section{Parcour}


\subsection{Dipômes}



\cventry{2018}{BAC}{S (SI) euroéen mention TB}
{Lycée Jules Ferry (Cannes)}{}{}

\cventry{2021}{Double Licence}{Maths / Info}
{Lyon 1 (Lyon)}{}{}

\subsection{Stages}

\cventry{2021}{stage}{INRIA}
{Bordeaux}{6 semaines}{couches minces}

\cventry{2014}{stage}{Cochlear}
{Bruxelles}{1 semaines}{Intoduction a la recherche}

\subsection{Cursus}


\cvline{2020-}{étudient ENS (Lyon)}
\cvline{2018-2020}{MPSI, MP* Lycée St-Louis (Paris)}{}{}


\section{Langue}
\cvlanguage{Français}{langue mère}{}
\cvlanguage{Anglais}{B2 / C1}{préparation du CAE}

\section{Langage de programmation}
\cvlistdoubleitem{JAVA}{C / C++}
\cvlistdoubleitem{Python}{Ocaml}



\section{Projets personnels (\href{https://github.com/Pierre-HB}{Github})}
\cvdoubleitem{2018}{Jeu 2D en JAVA, avec LWJGL}{2019}{Premier moteur 3D en Python}
\cvdoubleitem{2019/2020}{Meilleur moteur 3D en Python et Ocaml}{2020}{Moteur 3D utilisant OpenGL en JAVA}
\cvdoubleitem{2020}{Premier algorithme de ray tracing en JAVA}{2020}{Programme glouton pour le voyageur de commerce très optimisé en C++}

\section{Points personels}

\subsection{Centres d'interets}
\cvdoubleitem{}{Danse Rock}{}{Alpinisme}

\subsection{Qualitées}
\cvdoubleitem{}{Autonome}{}{Polyvalent}


\end{document}