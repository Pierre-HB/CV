\documentclass[11pt,a4paper]{moderncv}
\usepackage{multicol}
% moderncv themes
\moderncvtheme[green]{casual}
% \moderncvtheme[green]{classic}


% character encoding
\usepackage[utf8]{inputenc}
% adjust the page margins
% \usepackage[scale=0.8]{geometry}
\usepackage[top=2cm,bottom=2.5cm,left=2cm,right=2cm]{geometry}
\recomputelengths
% personal data
\firstname{Pierre}
\familyname{Hubert-Brierre}
\title{Doctorant en informatique graphique}
\address{91400 Orsay}{France}
\phone{+33 (0) 6 63 31 76 25}
\email{pierre.hubert-brierre@ens-lyon.fr}
\extrainfo{\url{https://github.com/Pierre-HB}}
\photo[64pt]{photo_Pierre.jpg}
\begin{document}



\maketitle

\vspace{1cm}
\section{Publications}
\cventry{2025}{Accelerating Signed Distance Functions}{Pacific Graphics}{}{}{P. Hubert-Brierre, E. Guérin, A. Peytavie, E. Galin}

\vspace{0.5cm}
\section{Diplômes}


\cventry{2024}{Diplôme de l'ENS}{Informatique}
{ENS (Lyon)}{Master en recherche informatique}{}

\cventry{2023}{Master ID3D}{Informatique}
{Lyon 1 (Lyon)}{Master en informatique graphique}{}

\cventry{2022}{Cambridge Advanced}{Anglais niveau C1}
{}{}{}

\cventry{2021}{Licence}{Mathématique}{Lyon 1 (Lyon)}{}{}
\cventry{2021}{Licence}{Informatique}{Lyon 1 (Lyon)}{}{}

\cventry{2018}{BAC}{S (SI) européen mention TB}
{Lycée Jules Ferry (Cannes)}{}{}


\vspace{0.5cm}
\section{Enseignements}

\cventry{2025}{Introduction OpenGL}{Polytechnique (36h) M1}
{Assistant professeur}{}{}
\cventry{2024/2025}{Vision assisté par ordinateur}{Polytechnique (12h+18h) M2}
{Assistant professeur}{}{}

\vspace{0.5cm}
\section{Cursus}


\cvline{2024-}{Doctorant Informatique (Lyon 1/Polytechnique)}
\cvline{2020-2024}{Normalien élève ENS (Lyon)}
\cvline{2023-2024}{étudiant PLR ENS (Lyon)}
\cvline{2022-2023}{étudiant ID3D (Lyon) (master d'infographie)}
\cvline{2021-2022}{étudiant master ENS (Lyon)}
\cvline{2020-2021}{étudiant licence ENS (Lyon)}
\cvline{2018-2020}{MPSI, MP* Lycée St-Louis (Paris)}{}{}

\newpage
\section{Stages}


\cventry{2024}{LIX}{Polytechnique}{5 mois, avec Marie-Paule CANI}
{\newline Optimisation de surfaces implicites}{\begin{itemize} \item Stage préparatoire pour mon sujet de thèse.\end{itemize}}

\cventry{2023}{LIRIS}{Lyon 1}{3 mois, avec Eric GALIN}
{\newline Modélisation par surfaces implicites}{\begin{itemize} \item Stage préparatoire pour mon sujet de thèse.\item Trois découverte majeurs.\end{itemize}}

\cventry{2023}{LIRIS}{Lyon 1}{5 mois, avec Jean-Claude IHEL}
{\newline Échantillonnage défensif pour le rendu Monte Carlo}{\begin{itemize} \item Deux découverte mineure.\end{itemize}}

\cventry{2022}{University of Edimburg}{Édimbourg}{3 mois, avec Kartic SUBR}
{\newline NLP pour l'infographie}{\begin{itemize} \item Génération automatique de scene 3D.\item Étude utilisateur.\end{itemize}}

\cventry{2021}{INRIA}{Bordeaux}{6 semaines, avec Pascal BARLA}
{\newline Modélisation de couches minces}{\begin{itemize} \item Développement d'un logiciel pour précalculer des BRDF.\item Ce logiciel a part la suite été réutilisé.\end{itemize}}

%stage de 3eme...
%\cventry{2014}{Cochlear}{Bruxelles}{1 semaines, avec Florent HUBERT-BRIERRE}{\newline Introduction à la recherche}{}




\section{Projets personnels (\href{https://github.com/Pierre-HB}{https://github.com/Pierre-HB})}


\cvitem{2022}{Création procédurale de ville reliées par un réseau routier C++}
\cvitem{2022}{Simplification de surface implicite en C++}
\cvitem{2020}{Programme glouton pour le voyageur de commerce très optimisé en C++}
\cvitem{2020}{Premier algorithme de ray tracing en JAVA}
\cvitem{2020}{Moteur 3D utilisant OpenGL en JAVA}
\cvitem{2019/2020}{Meilleur moteur 3D en Python et Ocaml}
\cvitem{2019}{Premier moteur 3D en Python}
\cvitem{2018}{Jeu 2D en JAVA, avec LWJGL}

\section{Langues}


\cvlanguage{Français}{langue maternelle}{}
\cvlanguage{Anglais}{C1 (Cambridge Advanced)}{}



\section{Langages de programmation}

\vspace{0.2cm}

\cvlistdoubleitem{Python}{C / C++}

\section{Points personnels}

\subsection{Centres d'intérêts}


\cvlistdoubleitem{Jeux vidéos}{Danse rock}
\cvlistdoubleitem{Alpinisme}{Roller}

\vspace{0.2cm}
\subsection{Qualités}


\cvlistdoubleitem{Polyvalent}{Autonome}


\end{document}


